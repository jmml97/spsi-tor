\documentclass[
  a4paper,
  12pt,
  spanish,
]{scrartcl}

% Párrafos
\setlength{\parindent}{18pt}

%-------------------------------------------------------------------------------
%	PAQUETES
%-------------------------------------------------------------------------------

% Idioma

\usepackage[es-noindentfirst]{babel}

% Matemáticas

\usepackage{amsmath, amsthm, amssymb}
\usepackage{mathtools}
\usepackage{commath}

% Fuentes personalizadas para utilizar con XeLaTeX o LuaLaTeX

\usepackage[no-math]{fontspec}
\setmainfont{Libertinus Serif}
\setsansfont{Libertinus Sans}
\setmonofont[Scale=.9]{Libertinus Mono}

\usepackage[math-style=TeX]{unicode-math}
\setmathfont{Libertinus Math}

% Configuración de microtype

\defaultfontfeatures{Ligatures=TeX,Numbers=Lining}
\usepackage[activate={true,nocompatibility},final,tracking=true,factor=1100,stretch=10,shrink=10]{microtype}
\SetTracking{encoding={*}, shape=sc}{0}

% Enlaces y colores

\usepackage{hyperref}
\usepackage[dvipsnames]{xcolor}
\definecolor{webgreen}{rgb}{0,0.5,0}
\hypersetup{
  colorlinks=true,
  citecolor=webgreen,
  urlcolor=Maroon,
  linkcolor=RoyalBlue
}

% Otros elementos de página

\usepackage{enumitem}
\setlist[enumerate]{leftmargin=*, itemsep=0pt}
\setlist[itemize]{leftmargin=*, itemsep=0pt}

\usepackage[labelfont=sc]{caption}

% Tikz

\usepackage{tikz}
\usetikzlibrary{babel}
\usepackage{float}

% Código

\usepackage{listings}
\lstset{
	basicstyle=\footnotesize\ttfamily,%
	breaklines=true,%
	captionpos=b,                    % sets the caption-position to bottom
  tabsize=2,	                   % sets default tabsize to 2 spaces
  frame=lines,
  numbers=left,
  stepnumber=1,
  aboveskip=12pt,
  showstringspaces=false,
}
\renewcommand{\lstlistingname}{Listado}

\usepackage{fancyvrb}

% Bibliografía

\usepackage[sorting=none, style=apa, isbn=true]{biblatex}
\DefineBibliographyStrings{spanish}{
  urlseen = {Consultado},
  retrieved = {Consultado},
}
\addbibresource{bibliografia.bib}

% Lorem ipsum

\usepackage{blindtext}

% Márgenes
\usepackage[bottom=3.125cm, top=2.5cm, left=4.5cm, right=4.5cm, marginparwidth=70pt]{geometry}

% Fuentes

\usepackage{textcase}

\newfontfamily{\sacshape}{Libertinus Serif}[
  WordSpace={1.8},
  LetterSpace={18.0}
]

\newfontfamily{\slscshape}{Libertinus Serif}[
  WordSpace={1.8},
  LetterSpace={6.0}
]

\DeclareRobustCommand{\spacedallcaps}[1]{{\linespread{1.3}\sacshape\MakeTextUppercase{#1}}}% WordSpace=1.8
\DeclareRobustCommand{\spacedlowsmallcaps}[1]{{\slscshape\MakeTextLowercase{#1}}}% WordSpace=1.8

% Cabeceras de sección

\RedeclareSectionCommands[beforeskip=-3ex,
afterskip=2ex]{section,subsection,subsubsection}
%\addtokomafont{section}{\normalfont\large\spacedallcaps}
%\setkomafont{section}{\normalfont\large\scshape}
\RedeclareSectionCommand[beforeskip=-9ex, font=\normalfont\large\scshape, tocentryformat=\normalfont\scshape]{section}
\addtokomafont{subsection}{\normalfont\normalsize\itshape}
\RedeclareSectionCommand[beforeskip=-6ex,tocentryformat=\normalfont\itshape]{subsection}
\addtokomafont{subsubsection}{\normalfont}
\RedeclareSectionCommand[beforeskip=-4ex]{subsubsection}
\addtokomafont{paragraph}{\normalfont\itshape}
%-------------------------------------------------------------------------------
%	TÍTULO
%-------------------------------------------------------------------------------

\newcommand{\horrule}[1]{\rule{\linewidth}{#1}}

%-------------------------------------------------------------------------------
%	CONTENIDO
%-------------------------------------------------------------------------------

\begin{document}

\begin{titlepage}
  \vspace*{4cm}

  \begin{flushleft}
    \Huge
    \spacedallcaps{Red Tor}
    \horrule{2pt}
  \end{flushleft}

  \vspace{2em}

  \begin{flushright}
    \large
    José María Martín Luque\\
    Antonio Martín Ruiz\\
    Daniel Pozo Escalona\vspace{1em}
  
    \textit{Seguridad y Protección de Sistemas Informáticos}
  
    Grado en Ingeniería Informática
  
    \textsc{Universidad de Granada}\vspace{1em}
  
    \today\vspace{.5em}
  \end{flushright}
\end{titlepage}

\newpage

{\hypersetup{hidelinks}
\tableofcontents
}

\newpage

\section{Introducción: ¿qué es Tor?}
% Común

% Tor is a distributed overlay network designed to anonymize
% low-latency TCP-based applications such as web browsing, secure shell,
% and instant messaging. Clients choose a path through the network and
% build a ``circuit'', in which each node (or ``onion router'' or ``OR'')
% in the path knows its predecessor and successor, but no other nodes in
% the circuit.  Traffic flowing down the circuit is sent in fixed-size
% ``cells'', which are unwrapped by a symmetric key at each node (like
% the layers of an onion) and relayed downstream.

Tor es una red superpuesta\footnote{
  Una red superpuesta es una red virtual de nodos enlazados lógicamente que utiliza la infraestructura de otra red existente \parencite[154]{kurose_computer_2013}.
} distribuida diseñada para anonimizar aplicaciones de baja latencia basadas en el protocolo TCP, como pueden ser un navegador web, una aplicación que implemente el protocolo SSH o un servicio de mensajería instantánea.
Los clientes escogen un camino a través de la red y crean un <<circuito>> en el que cada nodo ---o <<\textit{router} cebolla>>--- en dicho camino conoce su predecesor y sucesor pero no el resto de nodos del circuito.

\section{Estructura de la red Tor}
% Antonio

\section{Enrutamiento cebolla}
% Antonio

\section{Algoritmos de cifrado}
% Jose

Las conexiones entre dos repetidores en un circuito Tor, o entre el cliente y un repetidor utilizan el protocolo TLS/SSLv3 para la autentificación y el cifrado de los enlaces. 
Se conoce como \textit{suite de cifrado} al conjunto de algoritmos utilizados al establecer una comunicación mediante TLS. 
Dicha \textit{suite} suele incluir un algoritmo de intercambio de intercambio de claves, un algoritmo de cifrado en bloque y un algoritmo de código de autenticación de mensaje. 
La especificación de Tor \parencite{dingledine_tor_2019} permite que el usuario elija una \textit{suite} de cifrado de entre una lista predefinida, aunque todas las implementaciones deben soportar como mínimo \texttt{TLS\_DHE\_RSA\_WITH\_AES\_128\_CBC\_SHA}.

Sin entrar en demasiados detalles, el protocolo TLS tiene dos fases: una primera en la que ambas partes establecen la conexión y se ponen de acuerdo en una clave compartida con la que cifrar la información (mediante el uso del algoritmo de intercambio de claves), y una segunda en la que proceden al intercambio de información en sí, utilizando para ello el algoritmo de cifrado en bloque \parencite{ibm_overview_2019}.

Para establecer la conexión TLS entre los nodos del circuito creado por el usuario se utiliza una implementación del protocolo de Diffie-Hellman. 
Originalmente Tor utilizaba un protocolo de autentificación ---denominado posteriormente como TAP, \textit{Tor Authentication Protocol}--- que implementaba Diffie-Hellman sobre \(Z_p\), junto a RSA para calcular un conjunto de claves que ambos nodos comparten \parencite{dingledine_tor:_2004}. 

Análisis posteriores \parencite{hutchison_security_2006} llegaron a la conclusión de que TAP tenía deficiencias, lo que condujo a la creación de un nuevo protocolo denominado \textit{ntor} y que se introdujo en la versión \texttt{0.2.4.8-alpha} de Tor. 
Este nuevo protocolo utiliza ECDH, que implementa el protocolo  de Diffie-Hellman para un grupo curvas elípticas, en lugar de un grupo \(Z_p\). Tor en concreto utiliza el grupo Curve25519 \parencite{yung_curve25519:_2006}, que ofrece tamaños de clave de 128 bits.

Finalmente, una vez establecida la conexión TLS, se utiliza AES con claves de 128 bits para el cifrado de la información intercambiada entre dos nodos.

Como función \textit{hash}, se utiliza SHA-1 por defecto, aunque también se utilizan SHA256 y SHA3-256 en algunos puntos.

\subsection{Parámetros de los algoritmos}

La especificación de Tor describe los parámetros que se utilizan por defecto cuando se usa alguno de los algoritmos de cifrado descritos anteriormente, a saber: \begin{itemize}
  \item Para Diffie-Hellman sobre \(Z_p\) se utiliza como generador \(g=2\) y \(p\) es el primo seguro de 1024 bits descrito en \parencite{carrel_internet_1998}, cuyo valor es \[
    2^{1024} - 2^{960} - 1 + 2^{64} \cdot \left\lfloor (2^{894} \pi) + 129093 \right\rfloor.
  \] 
  %y cuya representación hexadecimal se describe en la figura \ref{verb:primo-seguro}.
  % \begin{figure}[h]
  %   \centering
  %   \begin{BVerbatim}
  % FFFFFFFF FFFFFFFF C90FDAA2 2168C234 C4C6628B
  % 80DC1CD1 29024E08 8A67CC74 020BBEA6 3B139B22
  % 514A0879 8E3404DD EF9519B3 CD3A431B 302B0A6D
  % F25F1437 4FE1356D 6D51C245 E485B576 625E7EC6
  % F44C42E9 A637ED6B 0BFF5CB6 F406B7ED EE386BFB
  % 5A899FA5 AE9F2411 7C4B1FE6 49286651 ECE65381
  % FFFFFFFF FFFFFFFF
  %   \end{BVerbatim}
  %   \caption{Representación hexadecimal del primo}
  %   \label{verb:primo-seguro}
  % \end{figure}
  \item Para Diffie-Hellman con curvas elípticas se utiliza el grupo Curve25519, como ya se ha mencionado antes.
  \item Para RSA se utilizan claves de tamaño \(1024\) y un exponente fijo, \(65537\).
\end{itemize}

\section{Posibles ataques}
% Dani


%-------------------------------------------------------------------------------
%	BIBLIOGRAFÍA
%-------------------------------------------------------------------------------

\newpage
\printbibliography

\end{document}
